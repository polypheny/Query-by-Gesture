\section{Concepts}
\label{g1:sec:concepts} % TODO Adopt group prefix

As mentioned in the introduction the goal is to connect the two separate systems Deepmime and Polypheny to have an additional HMI for Polypheny, instead of only mouse and keyboard.

\subsection{Deepmime}
Deepmime is a gesture recognition implementation based on a 3D residual neural network (ResNet). A residual neural network is an artificial neural network (ANN) of a kind that builds on constructs known from pyramidal cells in the cerebral cortex.\footnote{\url{https://en.wikipedia.org/wiki/Residual_neural_network}} For the implementation to recognise gestures on the fly, the skip connection is the relevant aspect for choosing a ResNet. The implementation we worked with takes a pre-trained model and classifies the video feed into a probability which labels it could be. 


\subsection{Polypheny}
"The last years have seen a vast diversification on the database market. In contrast to the “one-size-fits-all” paradigm according to which systems have been designed in the past, today’s database management systems (DBMS) are tuned for particular workloads. {[}...{]} In such cases, multistores are increasingly gaining popularity. Rather than supporting one single database paradigm and addressing one particular workload, multistores encompass several DBMSs that store data in different schemas and allow to route requests on a per-query level to the most appropriate system.{[}...{]}"\footnote{Icarus: Towards a Multistore Database System (\url{https://edoc.unibas.ch/58210/})} 
\newline
\\
Polypheny addresses exactly this problem. For our project, we are not dependant on the polystore aspect of the database but on the relational algebra part. Polypheny supports a graphical user interface (Polypheny-UI) where SQL queries can be built, connected via drag and drop and ran afterwards. There the user must understand SQL or relational algebra to build such a tree, hence our implementation also does not address users which can not build statements.
